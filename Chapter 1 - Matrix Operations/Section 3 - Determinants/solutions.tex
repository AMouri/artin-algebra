\documentclass[12pt]{article}
\usepackage{amsmath, mathdots}
\begin{document}
\title{Chapter 1: Matrix Operations \\ Section 3: Determinants}
\author{Alec Mouri}

\maketitle
\section*{Exercises}
\begin{itemize}
\item[(1)]
\begin{itemize}
\item[(a)]
$$\det\begin{bmatrix}
1 & i \\
2 - i & 3
\end{bmatrix} = (1)(3) - (i)(2 - i) = 3 - 2i + i^2 = 2 - 2i$$
\item[(b)]
$$\det\begin{bmatrix}
1 & 1 \\
1 & -1
\end{bmatrix} = (1)(-1) - (1)(1) = -1 - 1 = -2$$
\item[(c)]
$$\det\begin{bmatrix}
2 & 0 & 1 \\
0 & 1 & 0 \\
1 & 0 & 2
\end{bmatrix} = 2\det\begin{bmatrix}
1 & 0 \\
0 & 2
\end{bmatrix} - 0\det\begin{bmatrix}
0 & 1 \\
0 & 2
\end{bmatrix} + 1\det\begin{bmatrix}
0 & 1 \\
1 & 0
\end{bmatrix}$$
$$= 2((1)(2) - (0)(0)) + 1((0)(0) - (1)(1)) = 4 - 1 = 3$$
\item[(d)]
$$\det\begin{bmatrix}
1 & 0 & 0 & 0 \\
5 & 2 & 0 & 0 \\
8 & 6 & 3 & 0 \\
0 & 9 & 7 & 4
\end{bmatrix} = \det\begin{bmatrix}
1 & 5 & 8 & 0 \\
0 & 2 & 6 & 9 \\
0 & 0 & 3 & 7 \\
0 & 0 & 0 & 4
\end{bmatrix} $$
$$= 1\det\begin{bmatrix}
2 & 6 & 9 \\
0 & 3 & 7 \\
0 & 0 & 4
\end{bmatrix} = 2\det\begin{bmatrix}
3 & 7 \\
0 & 4
\end{bmatrix} = 2((3)(4) - (7)(0)) = 24$$
\item[(e)]
$$\det\begin{bmatrix}
1 & 4 & 1 & 3 \\
2 & 3 & 5 & 0 \\
4 & 1 & 0 & 0 \\
2 & 0 & 0 & 0
\end{bmatrix} = -\det\begin{bmatrix}
3 & 4 & 1 & 1 \\
0 & 3 & 5 & 2 \\
0 & 1 & 0 & 4 \\
0 & 0 & 0 & 2 \\
\end{bmatrix} = \det\begin{bmatrix}
3 & 1 & 4 & 1 \\
0 & 5 & 3 & 2 \\
0 & 0 & 1 & 4 \\
0 & 0 & 0 & 2
\end{bmatrix}$$
$$= 3\det\begin{bmatrix}
5 & 3 & 2 \\
0 & 1 & 4 \\
0 & 0 & 2
\end{bmatrix} = 15\begin{bmatrix}
1 & 4 \\
0 & 2
\end{bmatrix} = 15(2 - 0) = 30$$
\end{itemize}
\item[(2)]
$$\det\begin{bmatrix}
1 & 2 & 5 & 6 \\
3 & 1 & 7 & 7 \\
0 & 0 & 2 & 3 \\
4 & 2 & 1 & 5
\end{bmatrix} = -\det\begin{bmatrix}
2 & 1 & 5 & 6 \\
1 & 3 & 7 & 7 \\
0 & 0 & 2 & 3 \\
2 & 4 & 1 & 5
\end{bmatrix}$$
$$= -\det\begin{bmatrix}
2 & 1 & 5 & 6-5 \\
1 & 3 & 7 & 7-7 \\
0 & 0 & 2 & 3-2 \\
2 & 4 & 1 & 5-1
\end{bmatrix} = -\det\begin{bmatrix}
2 & 1 & 5 & 1 \\
1 & 3 & 7 & 0 \\
0 & 0 & 2 & 1 \\
2 & 4 & 1 & 4
\end{bmatrix}$$
\item[(3)]
$$\det A = (2)(4) - (1)(3) = 5$$
$$\det B = (1)(-2) - (5)(1) = -7$$
$$\det AB = \det\begin{bmatrix}
17 & -4 \\
21 & -7
\end{bmatrix} = (17)(-7) - (21)(-4) = -119 + 84 = -35$$
$$\det AB = -35 = (5)(-7) = (\det A)(\det B)$$
\item[(4)]
\begin{itemize}
\item[(a)]
Let $A$ be an $n \times n$ matrix, where
$$A = \begin{bmatrix}
& & & & 1 \\
& & & 1 & \\
& & \iddots & & \\
& 1 & & & \\
1 & & & &
\end{bmatrix}$$
I claim that $\det A = 1$. 

Suppose $n = 1$. Then clearly $\det A = 1$. Suppose for $n = k - 1$, that $\det A = 1$. Then, if $n = k$,
$$A = \begin{bmatrix}
\begin{array}{c|c}
& A' \\
\hline
1 & 
\end{array}
\end{bmatrix}$$
Where $A'$ has dimensions $k - 1 \times k - 1$. Note that $\det A' = 1$, from the inductive hypothesis. Then
$$\det A = 1\det A' = 1$$
\item[(b)]
Let $A$ be an $n \times n$ matrix, where
$$A = \begin{bmatrix}
2 & -1 \\
-1 & 2 & -1 \\
& -1 & 2 & -1 \\
& & -1 & \ddots \\
& & & & 2 & -1 \\
& & & & -1 & 2
\end{bmatrix}$$
I claim that $\det A = n + 1$.

Suppose $n = 1$. Then clearly $\det A = 2$. Suppose for $n = i$, where $i < k$, that $\det A = i + 1$. Then, if $n = k$,
$$A = \begin{bmatrix}
\begin{array}{c|ccc}
2 & -1 & 0 & \cdots \\
\hline
-1 & \\
0 & & A' \\
\vdots &
\end{array}
\end{bmatrix}$$
Where $A'$ has dimensions $k - 1 \times k - 1$. In particular,
$$A' = \begin{bmatrix}
\begin{array}{c|ccc}
2 & -1 & 0 & \cdots \\
\hline
-1 & \\
0 & & A" \\
\vdots &
\end{array}
\end{bmatrix}$$
Where $A"$ has dimensions $k - 2 \times k - 2$. 
Note that $\det A' = k$, and $\det A" = k - 1$, from the inductive hypothesis. Then
$$\det A = 2\det A' - (-1)\det\begin{bmatrix}
\begin{array}{c|cc}
-1 & 0 & \cdots\\
\hline
-1 & \\
0 & & A" \\
\vdots &
\end{array}
\end{bmatrix}$$
$$= 2k + \det\begin{bmatrix}
\begin{array}{c|ccc}
-1 & -1 & 0 & \cdots \\
\hline
0 & \\
0 & & A"^\top \\
\vdots &
\end{array}
\end{bmatrix} = 2k - \det A"^\top$$
$$= 2k - \det A" = 2k - (k - 1) = k + 1$$
\end{itemize}
\item[(5)]
Lemma: Let $A$ be a $n \times n$ upper triangular matrix. Then $\det A = a_{11}a_{22}...a_{nn}$. Proof: If $A$ is a $1 \times 1$ matrix, then trivially $\det A = a_{11}$. Suppose the statement is true for all $k - 1 \times k - 1$ matrices. Suppose $A$ is a $k \times k$ matrix, ie.
$$A = \begin{bmatrix}
\begin{array}{c|c}
a_{11} & * \\
& A'
\end{array}
\end{bmatrix}$$
where $A'$ is a $k - 1 \times k - 1$ matrix, and $*$ is a $1 \times k - 1$ matrix. Then $\det A = a_{11}\det A' = a_{11}a_{22}...a_{nn}$.

Thus, via the Lemma,
$$\det\begin{bmatrix}
1 & 2 & 3 & \cdots & n \\
2 & 2 & 3 & & \vdots \\
3 & 3 & 3 & & \vdots \\
\vdots & & & \ddots & \vdots \\
n & \cdots & \cdots & \cdots & n
\end{bmatrix} = \det\begin{bmatrix}
-1 & 2 & 3 & \cdots & n \\
0 & 2 & 3 & & \vdots \\
0 & 3 & 3 & & \vdots \\
\vdots & & & \ddots & \vdots \\
0 & n & \cdots & \cdots & n
\end{bmatrix}$$
$$= ... = \det\begin{bmatrix}
-1 & -1 & -1 & \cdots & -1 & n \\
0 & -1 & -1 & \cdots & -1 & n \\
0 & 0 & -1 & \cdots & -1 & n \\
\vdots & & & \ddots & \vdots & \vdots \\
0 & 0 & \cdots & \cdots & 0 & n
\end{bmatrix} = (-1)^{n-1}n$$
\item[(6)]
$$\det\begin{bmatrix}
2 & 1 \\
1 & 2 & 1 \\
& 1 & 2 & 1 \\
& & 1 & 2 & 1 \\
& & & 1 & 2 & 1 & & 1\\
& & & & 1 & 2 & 1 \\
& & & & & 1 & 2 \\
& & & & 1 & & & 2
\end{bmatrix} = \frac{1}{2}\det\begin{bmatrix}
2 & 1 \\
2 & 4 & 2 \\
& 1 & 2 & 1 \\
& & 1 & 2 & 1 \\
& & & 1 & 2 & 1 & & 1\\
& & & & 1 & 2 & 1 \\
& & & & & 1 & 2 \\
& & & & 1 & & & 2
\end{bmatrix}$$
$$= \frac{1}{2}\det\begin{bmatrix}
2 & 1 \\
& 3 & 2 \\
& 1 & 2 & 1 \\
& & 1 & 2 & 1 \\
& & & 1 & 2 & 1 & & 1\\
& & & & 1 & 2 & 1 \\
& & & & & 1 & 2 \\
& & & & 1 & & & 2
\end{bmatrix} = \frac{1}{6}\det\begin{bmatrix}
2 & 1 \\
& 3 & 2 \\
& 3 & 6 & 3 \\
& & 1 & 2 & 1 \\
& & & 1 & 2 & 1 & & 1\\
& & & & 1 & 2 & 1 \\
& & & & & 1 & 2 \\
& & & & 1 & & & 2
\end{bmatrix}$$
$$= \frac{1}{6}\det\begin{bmatrix}
2 & 1 \\
& 3 & 2 \\
& & 4 & 3 \\
& & 1 & 2 & 1 \\
& & & 1 & 2 & 1 & & 1\\
& & & & 1 & 2 & 1 \\
& & & & & 1 & 2 \\
& & & & 1 & & & 2
\end{bmatrix} = \frac{1}{24}\det\begin{bmatrix}
2 & 1 \\
& 3 & 2 \\
& & 4 & 3 \\
& & 4 & 8 & 4 \\
& & & 1 & 2 & 1 & & 1\\
& & & & 1 & 2 & 1 \\
& & & & & 1 & 2 \\
& & & & 1 & & & 2
\end{bmatrix}$$
$$= \frac{1}{24}\det\begin{bmatrix}
2 & 1 \\
& 3 & 2 \\
& & 4 & 3 \\
& & & 5 & 4 \\
& & & 1 & 2 & 1 & & 1\\
& & & & 1 & 2 & 1 \\
& & & & & 1 & 2 \\
& & & & 1 & & & 2
\end{bmatrix} = \frac{1}{120}\det\begin{bmatrix}
2 & 1 \\
& 3 & 2 \\
& & 4 & 3 \\
& & & 5 & 4 \\
& & & 5 & 10 & 5 & & 5\\
& & & & 1 & 2 & 1 \\
& & & & & 1 & 2 \\
& & & & 1 & & & 2
\end{bmatrix}$$
$$= \frac{1}{120}\det\begin{bmatrix}
2 & 1 \\
& 3 & 2 \\
& & 4 & 3 \\
& & & 5 & 4 \\
& & & & 6 & 5 & & 5\\
& & & & 1 & 2 & 1 \\
& & & & & 1 & 2 \\
& & & & 1 & & & 2
\end{bmatrix}$$
$$= \frac{1}{4320}\det\begin{bmatrix}
2 & 1 \\
& 3 & 2 \\
& & 4 & 3 \\
& & & 5 & 4 \\
& & & & 6 & 5 & & 5\\
& & & & 6 & 12 & 6 \\
& & & & & 1 & 2 \\
& & & & 6 & & & 12
\end{bmatrix} = \frac{1}{4320}\det\begin{bmatrix}
2 & 1 \\
& 3 & 2 \\
& & 4 & 3 \\
& & & 5 & 4 \\
& & & & 6 & 5 & & 5\\
& & & & & 7 & 6 & -5\\
& & & & & 1 & 2 \\
& & & & & -5 & & 7
\end{bmatrix}$$
$$= \frac{1}{5292000}\det\begin{bmatrix}
2 & 1 \\
& 3 & 2 \\
& & 4 & 3 \\
& & & 5 & 4 \\
& & & & 6 & 5 & & 5\\
& & & & & 35 & 30 & -25\\
& & & & & 35 & 70 \\
& & & & & -35 & & 49
\end{bmatrix}$$
$$= \frac{1}{5292000}\det\begin{bmatrix}
2 & 1 \\
& 3 & 2 \\
& & 4 & 3 \\
& & & 5 & 4 \\
& & & & 6 & 5 & & 5\\
& & & & & 35 & 30 & -25\\
& & & & & & 40 & 25\\
& & & & & & 30 & 24
\end{bmatrix}$$
$$= \frac{1}{63504000}\det\begin{bmatrix}
2 & 1 \\
& 3 & 2 \\
& & 4 & 3 \\
& & & 5 & 4 \\
& & & & 6 & 5 & & 5\\
& & & & & 35 & 30 & -25\\
& & & & & & 120 & 75\\
& & & & & & 120 & 96
\end{bmatrix}$$
$$= \frac{1}{63504000}\det\begin{bmatrix}
2 & 1 \\
& 3 & 2 \\
& & 4 & 3 \\
& & & 5 & 4 \\
& & & & 6 & 5 & & 5\\
& & & & & 35 & 30 & -25\\
& & & & & & 120 & 75\\
& & & & & & & 21
\end{bmatrix}$$
From the Lemma of the previous exercise, then we have
$$\frac{1}{63504000}(2\cdot 3 \cdot 4 \cdot 5 \cdot 6 \cdot 35 \cdot 120 \cdot 21) = 1$$
\item[(7)]
Suppose $A$ is a $1 \times 1$ matrix. Then for an arbitrary $1 \times 1$ matrix $B$, $\det(A + B) = a_{11} + b_{11} = \det A + \det B$. Furthermore, if $C = [ca_{11}]$, then $\det C = ca_{11} = c\det A$.

Suppose for all $n - 1 \times n - 1$ matrices, the determinant operates linearly on rows. Consider the $n \times n$ matrices $A, B, C$, where for some $k$, $c_{kj} = a_{kj} + b_{kj}$. Otherwise, if $i \neq k$, $a_{ij} = b_{ij} = c_{ij}$. In particular, $A_{i1} = B_{i1} = C_{i1}$, and by the inductive hypothesis $\det C_{k1} = \det A_{k1} + \det B_{k1}$. Then
$$\det C = \sum_{i=1}^n (-1)^{i+1}c_{i1}\det C_{i1}$$
$$= (-1)^{k+1}(a_{k1} + b_{k1})\det A_{k1} + \sum_{i \neq k}(-1)^{i+1}a_{i1}(\det A_{i1} + \det B_{i1})$$
$$= \sum_{i=1}^n (-1)^{i+1}a_{i1}\det A_{i1} + \sum_{i=1}^n (-1)^{i+1} b_{i1} \det B_{i1} = \det A + \det B$$

Now consider the $n \times n$ matrices $A', B'$, where for some $k$, $b_{kj} = ca_{kj}$, Otherwise, if $i \neq k$, $a_{ij} = b_{ij}$. In particular, $A_{i1} = B_{i1}$, and by the inductive hypothesis $\det B_{k1} = c\det A_{k1}$. Then
$$\det B = \sum_{i=1}^n (-1)^{i+1}b_{i1}\det B_{i1}$$
$$= (-1)^{k+1}ca_{kj}\det A_{k1} + \sum_{i \neq k}(-1)^{i + 1}a_{i1}(c\det A_{k1})$$ 
$$= c\sum_{i=1}^n (-1)^{k+1}a_{i1}\det A_{i1} = c\det A$$
\item[(8)]
$$\det(-A) = \det\begin{bmatrix}
-A_1 \\
\hline
-A_2 \\
\hline
\vdots \\
\hline
-A_n
\end{bmatrix} = (-1)^n\det\begin{bmatrix}
A_1 \\
\hline
A_2 \\
\hline
\vdots \\
\hline
A_n
\end{bmatrix} = (-1)^{n}\det A$$
\item[(9)]
Lemma: Let $E$ be an elementary matrix. Then $\det E = \det E^\top$. Proof: If $E = I + ae_{ij}$ is an elementary matrix of the first kind, then $E^\top = I + ae_{ji}$. Clearly, $E^\top$ is also an elementary matrix of the first kind, so $\det E = \det E^\top = 1$. If $E = I + e_{ij} + e_{ji} - e_{ii} - e_{jj}$ is an elementary matrix of the second kinda, then $E^\top = I + e_{ji} + e_{ij} - e_{ii} - e_{jj} = E$, so $\det E = \det E^\top$. If $E = I + (c - 1)e_{ii}$ is an elementary matrix of the third kind, then $E^\top = I +  (c - 1)e_{ii} = E$, so $\det E = \det E^\top$. 

Suppose $A$ is not invertible. Then $A^\top$ is also not invertible, and therefore $\det A = \det A^\top = 0$.

Suppose $A$ is invertible. Then $A^\top$ is also invertible, and for some elementary matrices $E_1, ..., E_p$, $\det E_p...\det E_1\det A = \det A^\top \det E_1^\top ... \det E_p^\top = \det I = 1$. Note that from the Lemma, $\det E_i = \det E_i^\top$. Therefore, $\det A = \det A^\top$.
\item[(10)]
$$\det\begin{bmatrix}
a & b \\
c & d
\end{bmatrix} = \det\begin{bmatrix}
a & 0 \\
c & d
\end{bmatrix} + \det\begin{bmatrix}
0 & b \\
c & d
\end{bmatrix}$$
$$ad\det\begin{bmatrix}
1 & 0 \\
c/d & 1
\end{bmatrix} - \det\begin{bmatrix}
c & d \\
0 & b
\end{bmatrix} = ad\det\begin{bmatrix}
1 & 0 \\
c/d & 1
\end{bmatrix} - cb\det\begin{bmatrix}
1 & d/c \\
0 & 1
\end{bmatrix}$$
$$= ad\left(\det\begin{bmatrix}
1 & 0 \\
0 & 1
\end{bmatrix} + \det\begin{bmatrix}
1 & 0 \\
c/d & 0
\end{bmatrix}\right) - bc\left(\det\begin{bmatrix}
1 & 0 \\
0 & 1
\end{bmatrix} + \det\begin{bmatrix}
0 & d/c \\
0 & 1
\end{bmatrix}\right)$$
$$= ad\left(1 + \frac{c}{d}\det\begin{bmatrix}
1 & 0 \\
1 & 0
\end{bmatrix}\right) - bc\left(1 + \frac{d}{c}\det\begin{bmatrix}
0 & 1 \\
0 & 1
\end{bmatrix}\right) = ad - bc$$
\item[(11)]
$$\det(AB) = (\det A)(\det B) = (\det B)(\det A) = \det(BA)$$
\item[(12)]
Suppose $A$ is a $1 \times 1$ submatrix. Then trivially
$$\det\begin{bmatrix}
A & B \\
0 & D
\end{bmatrix} = (\det A)(\det D)$$
Suppose the statement is true for all $k - 1 \times k - 1$ submatrices $A$. Let $A'$ be a $k \times k$ submatrix. Let 
$$X = \begin{bmatrix}
A' & B' \\
0 & D'
\end{bmatrix}$$
Then
$$\det X = \sum_{i=1}^k (-1)^{i+1}a'_{i1}\det X_{i1} = \sum_{i=1}^k (-1)^{i+1}a'_{i1}(\det A'_{i1})(\det D)$$
$$= (\det D)\sum_{i=1}^k (-1)^{i+1}a'_{i1}(\det A'_{i1}) = (\det D)(\det A')$$
\item[(13)]
Let
$$X = \begin{bmatrix}
I_n \\
-C & A
\end{bmatrix}, Y = \begin{bmatrix}
A & B \\
C & D
\end{bmatrix}$$
Then
$$XY = \begin{bmatrix}
A & B \\
-CA + AC & -CB + AD
\end{bmatrix} = \begin{bmatrix}
A & B \\
& AD - CB
\end{bmatrix}$$
Note that $\det X = \det X^\top = \det A$ and  $(\det X)(\det Y) = \det XY = (\det A)(\det (AD - CB))$. Thus $\det Y = \det(AD - CB)$.

Let
$$A = \begin{bmatrix}
1 & 1 \\
0 & 1
\end{bmatrix}, B = \begin{bmatrix}
1 & 0 \\
1 & 0
\end{bmatrix}, C = \begin{bmatrix}
1 & 0 \\
0 & 0
\end{bmatrix}, D = \begin{bmatrix}
1 & 0 \\
1 & 0
\end{bmatrix}$$
Note that $BC \neq CB$.
$$AD - CB = \begin{bmatrix}
2 & 0 \\
1 & 0
\end{bmatrix} - \begin{bmatrix}
1 & 1 \\
0 & 0
\end{bmatrix} = \begin{bmatrix}
1 & -1 \\
1 & 0
\end{bmatrix}$$
Note that $\det Y = 0$, but $\det (AD - CB) = 1$. Thus the formula does not hold.
\end{itemize}
\end{document}