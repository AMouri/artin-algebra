\documentclass[12pt]{article}
\usepackage{amsmath}
\begin{document}
\title{Chapter 1: Matrix Operations \\ Section 1: `The Basic Operations}
\author{Alec Mouri}

\maketitle
\section*{Exercises}
\begin{itemize}
\item[(1)]
$a_{21} = 2, s_{23} = 8$
\item[(2)]
\begin{itemize}
\item[(a)]
$$AB = \begin{bmatrix}
1 & 2 & 3 \\
3 & 3 & 1
\end{bmatrix}\begin{bmatrix}
-8 & -4 \\
9 & 5 \\
-3 & -2
\end{bmatrix} = \begin{bmatrix}
1 & 0 \\
0 & 1
\end{bmatrix}$$
$$BA = \begin{bmatrix}
-8 & -4 \\
9 & 5 \\
-3 & -2
\end{bmatrix}\begin{bmatrix}
1 & 2 & 3 \\
3 & 3 & 1
\end{bmatrix} = \begin{bmatrix}
-20 & -28 & - 28 \\
24 & 33 & 33 \\
-9 & -12 & -11
\end{bmatrix}$$
\item[(b)]
$$AB = \begin{bmatrix}
1 & 4 \\
1 & 2
\end{bmatrix}\begin{bmatrix}
6 & -4 \\
-3 & 2
\end{bmatrix} = \begin{bmatrix}
-6 & -4 \\
0 & 0
\end{bmatrix}$$
$$BA = \begin{bmatrix}
6 & -4 \\
-3 & 2
\end{bmatrix}\begin{bmatrix}
1 & 4 \\
1 & 2
\end{bmatrix} = \begin{bmatrix}
2 & 16 \\
-1 & -8
\end{bmatrix}$$
\item[(c)]
$$AB = \begin{bmatrix}
1 \\
-1 \\
0
\end{bmatrix}\begin{bmatrix}
1 & 2 & 1
\end{bmatrix} = \begin{bmatrix}
1 & 2 & 1 \\
-1 & -2 & -1 \\
0 & 0 & 0
\end{bmatrix}$$
$$BA = \begin{bmatrix}
1 & 2 & 1
\end{bmatrix}\begin{bmatrix}
1 \\
-1 \\
0
\end{bmatrix} = \begin{bmatrix}
-1
\end{bmatrix}$$
\end{itemize}
\item[(3)]
$$AB = \begin{bmatrix}
a_1 & \hdots & a_n
\end{bmatrix}\begin{bmatrix}
b_1 \\
\vdots \\
b_n
\end{bmatrix} = \begin{bmatrix}
a_1b_1 + ... + a_nb_n
\end{bmatrix}$$
$$BA = \begin{bmatrix}
b_1 \\
\vdots \\
b_n
\end{bmatrix}\begin{bmatrix}
a_1 & \hdots & a_n
\end{bmatrix} = \begin{bmatrix}
a_1b_1 & a_2b_1 & \hdots & a_nb_1 \\
a_1b_2 & a_2b_2 & \hdots & a_nb_2 \\
\vdots & \vdots & \ddots & \vdots \\
a_1b_n & a_2b_n & \hdots & a_nb_n
\end{bmatrix}$$
\item[(4)]
$$\left( \begin{bmatrix}
1 & 2 \\
0 & 1
\end{bmatrix}\begin{bmatrix}
0 & 1 & 2 \\
1 & 1 & 3
\end{bmatrix} \right)\begin{bmatrix}
1 \\
4 \\
3
\end{bmatrix} = \begin{bmatrix}
2 & 3 & 8 \\
1 & 1 & 3
\end{bmatrix}\begin{bmatrix}
1 \\
4 \\
3
\end{bmatrix} = \begin{bmatrix}
38 \\
14
\end{bmatrix}$$
$$\begin{bmatrix}
1 & 2 \\
0 & 1
\end{bmatrix} \left(\begin{bmatrix}
0 & 1 & 2 \\
1 & 1 & 3
\end{bmatrix}\begin{bmatrix}
1 \\
4 \\
3
\end{bmatrix} \right) = \begin{bmatrix}
1 & 2 \\
0 & 1
\end{bmatrix}\begin{bmatrix}
10 \\
14
\end{bmatrix} = \begin{bmatrix}
38 \\
14
\end{bmatrix}$$
\item[(5)]
$$\begin{bmatrix}
1 & a \\
& 1
\end{bmatrix}\begin{bmatrix}
1 & b \\
& 1
\end{bmatrix} = \begin{bmatrix}
1 & a + b \\
& 1
\end{bmatrix}$$
\item[(6)]
I claim that
$$\begin{bmatrix}
1 & 1 \\
& 1
\end{bmatrix}^n = \begin{bmatrix}
1 & n \\
& 1
\end{bmatrix}$$
Let $n = 1$. Then
$$\begin{bmatrix}
1 & 1 \\
& 1
\end{bmatrix}$$
So the statement is trivially true for $n = 1$. Suppose the statement is true for $n = k - 1$. Then
$$\begin{bmatrix}
1 & 1 \\
& 1
\end{bmatrix}^k = \begin{bmatrix}
1 & 1 \\
& 1
\end{bmatrix}^{k-1}\begin{bmatrix}
1 & 1 \\
& 1
\end{bmatrix} = \begin{bmatrix}
1 & k - 1 \\
& 1
\end{bmatrix}\begin{bmatrix}
1 & 1 \\
& 1
\end{bmatrix} = \begin{bmatrix}
1 & k \\
& 1
\end{bmatrix}$$
\item[(7)]
I claim that
$$\begin{bmatrix}
1 & 1 & 1 \\
& 1 & 1 \\
& & 1
\end{bmatrix}^n = \begin{bmatrix}
1 & n & T_n \\
& 1 & n \\
& & 1
\end{bmatrix}$$
Where $T_n = \sum_{i=1}^n i$

Suppose $n = 1$. Then
$$\begin{bmatrix}
1 & 1 & 1 \\
& 1 & 1 \\
& & 1
\end{bmatrix}^n = \begin{bmatrix}
1 & 1 & 1 \\
& 1 & 1 \\
& & 1
\end{bmatrix}$$
Since $T_1 = 1$, then the statement is true. Suppose the statement is true for $n = k - 1$. Then
$$\begin{bmatrix}
1 & 1 & 1 \\
& 1 & 1 \\
& & 1
\end{bmatrix}^k = \begin{bmatrix}
1 & 1 & 1 \\
& 1 & 1 \\
& & 1
\end{bmatrix}^{k-1}\begin{bmatrix}
1 & 1 & 1 \\
& 1 & 1 \\
& & 1
\end{bmatrix}$$
$$ = \begin{bmatrix}
1 & k-1 & T_{k-1} \\
& 1 & k-1 \\
& & 1
\end{bmatrix}\begin{bmatrix}
1 & 1 & 1 \\
& 1 & 1 \\
& & 1
\end{bmatrix} = \begin{bmatrix}
1 & k-1 + 1 & T_{k-1} + k-1 + 1 \\
& 1 & k -1 + 1 \\
& & 1 
\end{bmatrix}$$
$$= \begin{bmatrix}
1 & k & T_k \\
& 1 & k \\
& & 1
\end{bmatrix}$$
\item[(8)]
$$\begin{bmatrix} 
\begin{array}{cc|cc}
1 & 1 & 1 & 5 \\
0 & 1 & 0 & 1 \\
\hline
1 & 0 & 0 & 1 \\
0 & 1 & 1 & 0
\end{array}
\end{bmatrix}\begin{bmatrix} 
\begin{array}{cc|cc}
1 & 2 & 1 & 0 \\
0 & 1 & 0 & 1 \\
\hline
1 & 0 & 0 & 1 \\
0 & 1 & 1 & 3
\end{array}
\end{bmatrix}$$
$$= \begin{bmatrix}
\begin{array}{c|c}

\begin{bmatrix}
1 & 1 \\
0 & 1
\end{bmatrix} \begin{bmatrix}
1 & 2 \\
0 & 1
\end{bmatrix} + \begin{bmatrix}
1 & 5 \\
0 & 1
\end{bmatrix}\begin{bmatrix}
1 & 0 \\
0 & 1
\end{bmatrix} & \begin{bmatrix}
1 & 1 \\
0 & 1
\end{bmatrix} \begin{bmatrix}
1 & 0 \\
0 & 1
\end{bmatrix} + \begin{bmatrix}
1 & 5 \\
0 & 1
\end{bmatrix}\begin{bmatrix}
0 & 1 \\
1 & 3
\end{bmatrix} \\
\hline
\begin{bmatrix}
1 & 0 \\
0 & 1
\end{bmatrix} \begin{bmatrix}
1 & 2 \\
0 & 1
\end{bmatrix} + \begin{bmatrix}
0 & 1 \\
1 & 0
\end{bmatrix}\begin{bmatrix}
1 & 0 \\
0 & 1
\end{bmatrix} & \begin{bmatrix}
1 & 0 \\
0 & 1
\end{bmatrix} \begin{bmatrix}
1 & 0 \\
0 & 1
\end{bmatrix} + \begin{bmatrix}
0 & 1 \\
1 & 0
\end{bmatrix}\begin{bmatrix}
0 & 1 \\
1 & 3
\end{bmatrix}
\end{array}
\end{bmatrix}$$
$$= \begin{bmatrix}
\begin{array}{c|c}
\begin{bmatrix}
1 & 3 \\
0 & 1
\end{bmatrix} + \begin{bmatrix}
1 & 5 \\
0 & 1
\end{bmatrix} & \begin{bmatrix}
1 & 1 \\
0 & 1
\end{bmatrix} + \begin{bmatrix}
5 & 16 \\
1 & 3
\end{bmatrix} \\
\hline
\begin{bmatrix}
1 & 2 \\
0 & 1
\end{bmatrix} + \begin{bmatrix}
0 & 1 \\
1 & 0
\end{bmatrix} & \begin{bmatrix}
1 & 0 \\
0 & 1
\end{bmatrix} + \begin{bmatrix}
1 & 3 \\
0 & 1
\end{bmatrix}
\end{array}
\end{bmatrix} = \begin{bmatrix}
2 & 8 & 6 & 17 \\
0 & 2 & 1 & 4 \\
1 & 3 & 2 & 3 \\
1 & 1 & 0 & 1
\end{bmatrix}$$
$$\begin{bmatrix} 
\begin{array}{c|cc}
0 & 1 & 2 \\
\hline
0 & 1 & 0 \\
3 & 0 & 1
\end{array}
\end{bmatrix}\begin{bmatrix} 
\begin{array}{c|cc}
1 & 2 & 3 \\
\hline
4 & 2 & 3 \\
5 & 0 & 4
\end{array}
\end{bmatrix}$$
$$= \begin{bmatrix} 
\begin{array}{c|c}
\begin{bmatrix}
0
\end{bmatrix}\begin{bmatrix}
1
\end{bmatrix} + \begin{bmatrix}
1 & 2
\end{bmatrix}\begin{bmatrix}
4 \\
5
\end{bmatrix} & \begin{bmatrix}
0
\end{bmatrix}\begin{bmatrix}
2 & 3
\end{bmatrix} + \begin{bmatrix}
1 & 2
\end{bmatrix}\begin{bmatrix}
2 & 3 \\
0 & 4
\end{bmatrix} \\
\hline
\begin{bmatrix}
0 \\
3
\end{bmatrix}\begin{bmatrix}
1
\end{bmatrix} + \begin{bmatrix}
1 & 0 \\
0 & 1
\end{bmatrix}\begin{bmatrix}
4 \\
5
\end{bmatrix} & \begin{bmatrix}
0 \\
3
\end{bmatrix}\begin{bmatrix}
2 & 3
\end{bmatrix} + \begin{bmatrix}
1 & 0 \\
0 & 1
\end{bmatrix}\begin{bmatrix}
2 & 3 \\
0 & 4
\end{bmatrix}
\end{array}
\end{bmatrix}$$
$$= \begin{bmatrix}
\begin{array}{c|c}
\begin{bmatrix}
0
\end{bmatrix} + \begin{bmatrix}
14
\end{bmatrix} & \begin{bmatrix}
0 & 0
\end{bmatrix} + \begin{bmatrix}
2 & 11
\end{bmatrix} \\
\hline
\begin{bmatrix}
0 \\
3
\end{bmatrix} + \begin{bmatrix}
4 \\
5
\end{bmatrix} & \begin{bmatrix}
0 & 0 \\
6 & 9
\end{bmatrix} + \begin{bmatrix}
2 & 3 \\
0 & 4
\end{bmatrix}
\end{array}
\end{bmatrix} = \begin{bmatrix}
14 & 2 & 11 \\
4 & 2 & 3 \\
8 & 6 & 13
\end{bmatrix}$$
\item[(9)]
Let $M$ be a $m \times n$ matrix and $M'$ be a $n \times p$ matrix where 
$$M = \begin{bmatrix}
\begin{array}{c|c}
A & B \\
\hline
C & D
\end{array}
\end{bmatrix} = \begin{bmatrix}
\begin{array}{cccc|cccc}
a_{11} & a_{12} & \hdots & a_{1i} & b_{11} & b_{12} & \hdots & b_{1 n-i} \\
a_{21} & a_{22} & \hdots & a_{2i} & b_{21} & b_{22} & \hdots & b_{2 n-i} \\
\vdots & \vdots & \ddots & \vdots & \vdots & \vdots & \ddots & \vdots \\
a_{j1} & a_{j2} & \hdots & a_{ji} & b_{j1} & b_{j2} & \hdots & b_{j n-i} \\
\hline
c_{11} & c_{12} & \hdots & c_{1i} & d_{11} & d_{12} & \hdots & d_{1 n-i} \\
c_{21} & c_{22} & \hdots & c_{2i} & d_{21} & d_{22} & \hdots & d_{2 n-i} \\
\vdots & \vdots & \ddots & \vdots & \vdots & \vdots & \ddots & \vdots \\
c_{m-j 1} & c_{m-j 2} & \hdots & c_{m-j i} & d_{m-j 1} & d_{m-j 2} & \hdots & d_{m-j n-i} \\
\end{array}
\end{bmatrix}$$
$$M' = \begin{bmatrix}
\begin{array}{c|c}
A' & B' \\
\hline
C' & D'
\end{array}
\end{bmatrix} = \begin{bmatrix}
\begin{array}{cccc|cccc}
a'_{11} & a'_{12} & \hdots & a'_{1x} & b'_{11} & b'_{12} & \hdots & b'_{1 p-x} \\
a'_{21} & a'_{22} & \hdots & a'_{2x} & b'_{21} & b'_{22} & \hdots & b'_{2 p-x} \\
\vdots & \vdots & \ddots & \vdots & \vdots & \vdots & \ddots & \vdots \\
a'_{i1} & a'_{i2} & \hdots & a'_{ix} & b'_{i1} & b'_{i2} & \hdots & b'_{i p-x} \\
\hline
c'_{11} & c'_{12} & \hdots & c'_{1x} & d'_{11} & d'_{12} & \hdots & d'_{1 p-x} \\
c'_{21} & c'_{22} & \hdots & c'_{2x} & d'_{21} & d'_{22} & \hdots & d'_{2 p-x} \\
\vdots & \vdots & \ddots & \vdots & \vdots & \vdots & \ddots & \vdots \\
c'_{n-i 1} & c'_{n-i 2} & \hdots & c'_{n-i x} & d'_{n-i 1} & d'_{n-i 2} & \hdots & d'_{n-i p-x} \\
\end{array}
\end{bmatrix}$$
Then
$$MM' = \begin{bmatrix}
\sum_{k=1}^i a_{1k}a'_{k1} + \sum_{k=1}^{n-i} b_{1k}c'_{k1} & \hdots & \sum_{k=1}^i a_{1k}b'_{k p-x} + \sum_{k=1}^{n-i} b_{1k}d'_{k p-x} \\
\vdots & \ddots & \vdots \\
 \sum_{k=1}^i c_{1k}a'_{k1} + \sum_{k=1}^{n-i} d_{1k}c'_{k1} & \hdots & \sum_{k=1}^i c_{1k}b'_{k p-x} + \sum_{k=1}^{n-i} d_{1k}d'_{k p-x}
\end{bmatrix}$$
$$= \begin{bmatrix}
\begin{array}{c|c}
AA' + AC' & A'B + BD' \\
\hline
CA' + DC' & CB' + DD'
\end{array}
\end{bmatrix}$$
\item[(10)]
\begin{itemize}
\item[(a)]
$$A^2 - B^2 = (A + B)(A - B) = A^2 + BA - AB - B^2 \rightarrow BA = AB$$
\item[(b)]
$$(A + B)^3 = (A + B)(A^2 + AB + BA + B^2)$$
$$= A^3 + A^2B + ABA + AB^2 + BA^2 + BAB + B^2A + B^3$$
\end{itemize}
\item[(11)]
\begin{itemize}
\item[(a)]
$$DA = \begin{bmatrix}
d_1a_{11} & d_1a_{12} & \hdots & d_1a_{1n} \\
d_2a_{21}& d_2a_{22} & \hdots & d_2a_{2n} \\
\vdots & \vdots & \ddots & \vdots \\
d_na_{n1} & d_na_{n2} & \vdots & d_na_{nn}
\end{bmatrix}$$
$$AD = \begin{bmatrix}
d_1a_{11} & d_2a_{12} & \hdots & d_na_{1n} \\
d_1a_{21} & d_2a_{22} & \hdots & d_na_{2n} \\
\vdots & \vdots & \ddots & \vdots \\
d_1a_{n1} & d_2a_{n2} & \hdots & d_na_{nn}
\end{bmatrix}$$
\item[(b)]
Let $D, D'$ be $n \times n$ diagonal matrices with entries $d_{ii}$ and $d'_{ii}$ respectively. Then
$$DD' = \begin{bmatrix}
d_{11}d'_{11} & & & \\
& d_{22}d'_{22} & & \\
& & \ddots & \\
& & & d_{nn}d'_{nn}
\end{bmatrix}$$
\item[(c)]
From part (a), each diagonal entry $i$ of either product $DA$ or $AD$ equals $d_ia_{ii}$. If $A$ is an inverse of $D$, then each $d_ia_{ii} = 1$, implying each diagonal entry of $D$ must be nonzero.
\end{itemize}
\item[(12)]
Let $A, B$ be $n \times n$ upper triangular matrices. Then
$$(AB)_{ij} = \sum_{k=1}^n a_{ik}b_{kj} = \sum_{k=i}^j a_{ik}b_{kj}$$
Since $i > j$, then $(AB)_{ij} = 0$. Therefore, $AB$ is upper triangular.
\item[(13)]
\begin{itemize}
\item[(a)]
$$\begin{bmatrix}
1 & 0 \\
0 & 0
\end{bmatrix}\begin{bmatrix}
a & b \\
c & d
\end{bmatrix} = \begin{bmatrix}
a & b \\
c & d
\end{bmatrix}\begin{bmatrix}
1 & 0 \\
0 & 0
\end{bmatrix}$$
$$\rightarrow \begin{bmatrix}
a & b \\
0 & 0
\end{bmatrix} = \begin{bmatrix}
a & 0 \\
c & 0
\end{bmatrix}$$
Thus, a matrix that commutes has the form
$$\begin{bmatrix}
a & 0 \\
0 & 0
\end{bmatrix}$$
\item[(b)]
$$\begin{bmatrix}
0 & 1 \\
0 & 0
\end{bmatrix}\begin{bmatrix}
a & b \\
c & d
\end{bmatrix} = \begin{bmatrix}
a & b \\
c & d
\end{bmatrix}\begin{bmatrix}
0 & 1 \\
0 & 0
\end{bmatrix}$$
$$\rightarrow \begin{bmatrix}
c & d \\
0 & 0
\end{bmatrix} = \begin{bmatrix}
0 & a \\
0 & c
\end{bmatrix}$$
Thus, a matrix that commutes has the form
$$\begin{bmatrix}
0 & a \\
0 & 0
\end{bmatrix}$$
\item[(c)]
$$\begin{bmatrix}
2 & 0 \\
0 & 6
\end{bmatrix}\begin{bmatrix}
a & b \\
c & d
\end{bmatrix} = \begin{bmatrix}
a & b \\
c & d
\end{bmatrix}\begin{bmatrix}
2 & 0 \\
0 & 6
\end{bmatrix}$$
$$\rightarrow \begin{bmatrix}
2a & 2b \\
6c & 6d
\end{bmatrix} = \begin{bmatrix}
2a & 6b \\
2c & 6d
\end{bmatrix}$$
Thus, a matrix that commutes has the form
$$\begin{bmatrix}
a & 0 \\
0 & d
\end{bmatrix}$$
\item[(d)]
$$\begin{bmatrix}
1 & 3 \\
0 & 1
\end{bmatrix}\begin{bmatrix}
a & b \\
c & d
\end{bmatrix} = \begin{bmatrix}
a & b \\
c & d
\end{bmatrix}\begin{bmatrix}
1 & 3 \\
0 & 1
\end{bmatrix}$$
$$\rightarrow \begin{bmatrix}
a + 3c & b + 3d \\
c & d
\end{bmatrix} = \begin{bmatrix}
a & 3a + b \\
c & 3c + d
\end{bmatrix}$$
Thus, a matrix that commutes has the form
$$\begin{bmatrix}
a & b \\
0 & a
\end{bmatrix}$$
\item[(e)]
$$\begin{bmatrix}
2 & 3 \\
0 & 6
\end{bmatrix}\begin{bmatrix}
a & b \\
c & d
\end{bmatrix} = \begin{bmatrix}
a & b \\
c & d
\end{bmatrix}\begin{bmatrix}
2 & 3 \\
0 & 6
\end{bmatrix}$$
$$\rightarrow \begin{bmatrix}
2a + 3c & 2b + 3d \\
6c & 6d
\end{bmatrix} = \begin{bmatrix}
2a & 3a + 6b \\
2c & 3c + 6d
\end{bmatrix}$$
Thus, a matrix that commutes has the form
$$\begin{bmatrix}
a & \frac{3}{4}(d - a) \\
0 & d
\end{bmatrix}$$
\end{itemize}
\item[(14)]
Let $A = (a_{ij})$ be an $n \times n$ matrix. Then
$$0 + A = \begin{bmatrix}
0 & \hdots & 0 \\
\vdots & \ddots & \vdots \\
0 & \hdots & 0
\end{bmatrix} + \begin{bmatrix}
a_{11} & \hdots & a_{1n} \\
\vdots & \ddots & \vdots \\
a_{n1} & \hdots & a_{nn}
\end{bmatrix}$$
$$= \begin{bmatrix}
0 + a_{11} & \hdots & 0 + a_{1n} \\
\vdots & \ddots & \vdots \\
0 + a_{n1} & \hdots & 0 + a_{nn}
\end{bmatrix} = \begin{bmatrix}
a_{11} & \hdots & a_{1n} \\
\vdots & \ddots & \vdots \\
a_{n1} & \hdots & a_{nn}
\end{bmatrix} = A$$
$$0A = \begin{bmatrix}
0 & \hdots & 0 \\
\vdots & \ddots & \vdots \\
0 & \hdots & 0
\end{bmatrix}\begin{bmatrix}
a_{11} & \hdots & a_{1n} \\
\vdots & \ddots & \vdots \\
a_{n1} & \hdots & a_{nn}
\end{bmatrix}$$
$$ = \begin{bmatrix}
0a_{11} + \hdots + 0a_{n1} & \hdots & 0a_{1n} + \hdots + 0a_{nn} \\
\vdots & \ddots & \vdots \\
0a_{11} + \hdots + 0a_{n1} & \hdots & 0a_{1n} + \hdots + 0a_{nn}
\end{bmatrix} = \begin{bmatrix}
0 & \hdots & 0 \\
\vdots & \ddots & \vdots \\
0 & \hdots & 0
\end{bmatrix} = 0$$
$$A0 = \begin{bmatrix}
a_{11} & \hdots & a_{1n} \\
\vdots & \ddots & \vdots \\
a_{n1} & \hdots & a_{nn}
\end{bmatrix}\begin{bmatrix}
0 & \hdots & 0 \\
\vdots & \ddots & \vdots \\
0 & \hdots & 0
\end{bmatrix}$$
$$= \begin{bmatrix}
0a_{11} + \hdots + 0a_{1n} & \hdots & 0a_{11} + \hdots + 0a_{1n} \\
\vdots & \ddots & \vdots \\
0a_{n1} + \hdots + 0a_{nn} & \hdots & 0a_{n1} + \hdots + 0a_{nn}
\end{bmatrix} = \begin{bmatrix}
0 & \hdots & 0 \\
\vdots & \ddots & \vdots \\
0 & \hdots & 0
\end{bmatrix} = 0$$
\item[(15)]
Suppose an $n \times n$ $A$ matrix has a row $i$ of zeros. Then for any $n \times n$ matrix $B$,
$$(AB)_{ii} = \sum_{k=1}^n a_{ik}b_{ki} = 0$$
So $B$ cannot be an inverse of $A$, so $A$ is not invertible.
\item[(16)]
Suppose $k = 1$. Then $A = 0$, so trivially $I + A = I$ is invertible: its inverse is $I$.

Suppose $k > 1$. Consider $B = I + \sum_{i=1}^{k-1}(-1)^iA^i$. Then
$$(I+A)B = I + \sum_{i=1}^{k-1}\left((-1)^iA^i + (-1)^iA^{i+1}\right)$$
$$= I + (-1)^kA^k + \sum_{i=1}^{k-1}
\left( (-1)^{i-1}A^i + (-1)^iA^i \right) = I$$
Furthermore,
$$B(I+A) = I + \sum_{i=1}^{k-1}\left((-1)^iA^i + (-1)^iA^{i+1}\right)$$
$$= I + (-1)^kA^k + \sum_{i=1}^{k-1}
\left( (-1)^{i-1}A^i + (-1)^iA^i \right) = I$$
Thus, $(I + A)^{-1} = B$
\item[(17)]
\begin{itemize}
\item[(a)]
Consider the $2 \times 3$ matrix $B$ that is a left inverse of $A$, where
$$B = \begin{bmatrix}
a & b & c \\
d & e & f
\end{bmatrix}$$
Then
$$BA = \begin{bmatrix}
a & b & c \\
d & e & f
\end{bmatrix}\begin{bmatrix}
2 & 3 \\
1 & 2 \\
2 & 5
\end{bmatrix}$$
$$= \begin{bmatrix}
2a + b + 2c & 3a + 2b + 5c \\
2d + e + 2f & 3d + 2e + 5f
\end{bmatrix} = \begin{bmatrix}
1 & 0 \\
0 & 1
\end{bmatrix}$$
Thus
$$B = \begin{bmatrix}
2 + c & -3 - 4c & c \\
f - 1 & 2 - 4f & f
\end{bmatrix}$$
\item[(b)]
Consider the $2 \times 3$ matrix $C$, where
$$C = \begin{bmatrix}
a & b & c \\
d & e & f
\end{bmatrix}$$
Suppose $C$ is a right inverse of $A$. Then
$$AC = \begin{bmatrix}
2 & 3 \\
1 & 2 \\
2 & 5
\end{bmatrix}\begin{bmatrix}
a & b & c \\
d & e & f
\end{bmatrix}$$
$$ = \begin{bmatrix}
2a + 3d & 2b + 3e & 2c + 3f \\
a + 2d & b + 2e & c + 2f \\
2a + 5d & 2a + 5e & 2a + 5f
\end{bmatrix} = \begin{bmatrix}
1 & & \\
& 1 & \\
& & 1
\end{bmatrix} $$
In particular, $a + 2d = 2a + 5d = 0$ implies that $a = 0$ and $d = 0$. But then $0 = 2a + 3d = 1$, a contradiction. Thus $C$ cannot be a right inverse of $A$.
\end{itemize}
\item[(18)]
Assume $A, B$ are invertible. We can then check if $B^{-1}A^{-1}$ is the inverse of $AB$:
$$(AB)(B^{-1}A^{-1}) = A(BB^{-1})A^{-1} = A(I)A^{-1} = AA^{-1} = I$$
Similarly,
$$(B^{-1}A^{-1})(AB) = B^{-1}(A^{-1}A)B = B^{-1}(I)B = B^{-1}B = I$$
\item[(19)]
\begin{itemize}
\item[(a)]
Let $C = A + B$. Then $c_{ij} = a_{ij} + b_{ij}$. So
$$\text{tr }C = \sum_{i=1}^n c_{ii} = \sum_{i=1}^n a_{ii} + \sum_{i=1}^n b_{ii} = \text{tr }A + \text{tr }B$$
Let $D = AB$ and $E = BA$. Then $d_{ij} = \sum_{k=1}^n a_{ik}b_{kj}$, and $e_{ij} = \sum_{k=1}^n b_{ik}a_{kj}$. So
$$\text{tr }D = \sum_{i = 1}^n d_{ii} = \sum_{i=1}^n\sum_{k=1}^n a_{ik}b_{ki} = \sum_{k=1}^n\sum_{i=1}^n b_{ki}a_{ik} = \sum_{k=1}^n e_{kk} = \text{tr }E$$
\item[(b)]
Let $C = BA$. Then
$$\text{tr }BAB^{-1} = \text{tr }CB^{-1} \text{tr }B^{-1}C = \text{tr }B^{-1}BA = \text{tr }A$$
\end{itemize}
\item[(20)]
Note that $\text{tr }I = n$. Furthermore, for any matrix $A$ and a constant $c$, 
$$\text{tr }cA = ca_{11} + ca_{22} + ... + ca_{nn} = c(a_{11} + a_{22} + ... + a_{nn}) = c \cdot \text{tr }A$$ 
But
$$\text{tr }(AB - BA) = \text{tr }AB + \text{tr }(-1)BA = \text{tr }AB - \text{tr }BA = 0$$
Therefore, $AB - BA \neq I$.
\end{itemize}
\end{document}