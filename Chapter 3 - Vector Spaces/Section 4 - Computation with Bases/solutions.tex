\documentclass[12pt]{article}
\usepackage{amsmath, amssymb, mathdots}
\begin{document}
\title{Chapter 3: Vector Spaces \\ Section 4: Computation with Bases}
\author{Alec Mouri}

\maketitle
\section*{Exercises}
\begin{itemize}
\item[(1)]
$$E = B'P \rightarrow \begin{bmatrix}
1 & 0 \\
0 & 1
\end{bmatrix} = \begin{bmatrix}
1 & 2 \\
3 & 2
\end{bmatrix}\begin{bmatrix}
p_{11} & p_{12} \\
p_{21} & p_{22}
\end{bmatrix}$$
$$\rightarrow \begin{bmatrix}
1 & 0 \\
0 & 1
\end{bmatrix} = \begin{bmatrix}
p_{11} + 2p_{21} & p_{12} + 2p_{22} \\
3p_{11} + 2p_{21} & 3p_{12} + 2p_{22}
\end{bmatrix}$$
$$\rightarrow P = \begin{bmatrix}
-1/2 & 1/2 \\
3/4 & -3/4
\end{bmatrix}$$
\item[(2)]
$$I = BP \rightarrow P = B$$
$$\rightarrow P = \begin{bmatrix}
& & 1 \\
& \iddots \\
1
\end{bmatrix}$$
\item[(3)]
$$E = B'P \rightarrow \begin{bmatrix}
1 & 0 \\
0 & 1
\end{bmatrix} = \begin{bmatrix}
1 & 1 \\
1 & -1
\end{bmatrix}\begin{bmatrix}
p_{11} & p_{12} \\
p_{21} & p_{22}
\end{bmatrix}$$
$$\rightarrow \begin{bmatrix}
1 & 0 \\
0 & 1
\end{bmatrix}\begin{bmatrix}
p_{11} + p_{21} & p_{12} + p_{22} \\
p_{11} - p_{21} & p_{12} - p_{22}
\end{bmatrix}$$
$$\rightarrow P = \begin{bmatrix}
1/2 & 1/2 \\
1/2 & -1/2
\end{bmatrix}$$
\item[(4)]
$$E = B'P \rightarrow \begin{bmatrix}
1 & 0 \\
0 & 1
\end{bmatrix} = \begin{bmatrix}
1 & -1/2 \\
0 & \sqrt{3}/2
\end{bmatrix}\begin{bmatrix}
p_{11} & p_{12} \\
p_{21} & p_{22}
\end{bmatrix}$$
$$\rightarrow \begin{bmatrix}
1 & 0 \\
0 & 1
\end{bmatrix} = \begin{bmatrix}
p_{11} - p_{21}/2 & p_{12} - p_{22}/2 \\
\sqrt{3}p_{21}/2 & \sqrt{3}p_{22}/2
\end{bmatrix}$$
$$\rightarrow P = \begin{bmatrix}
1 & 1/\sqrt{3} \\
0 & 2/\sqrt{3}
\end{bmatrix}$$
\item[(5)]
\begin{itemize}
\item[(i)]
If we can find a $P$ such that $B = EP$, then $B$ is a basis for $\mathbb{R}^3$, since $E$ is a basis. Clearly, $$P = B = \begin{bmatrix}
1 & 2 & 3 \\
2 & 1 & 1 \\
0 & 2 & 1
\end{bmatrix}$$
\item[(ii)]
$$ \begin{bmatrix}
1 & 2 & 3 \\
2 & 1 & 1 \\
0 & 2 & 1
\end{bmatrix}\begin{bmatrix}
x_1 \\
x_2 \\
x_3
\end{bmatrix} = \begin{bmatrix}
1 \\
2 \\
3
\end{bmatrix}$$
$$\rightarrow \begin{bmatrix}
x_1 \\
x_2 \\
x_3
\end{bmatrix} = \begin{bmatrix}
-1/7 & 4/7 & -1/7 \\
-2/7 & 1/7 & 5/7 \\
4/7 & -2/7 & -3/7
\end{bmatrix}\begin{bmatrix}
1 \\
2 \\
3
\end{bmatrix} = \begin{bmatrix}
4/7 \\
15/7 \\
-9/7
\end{bmatrix}$$
\item[(iii)]
$$B' = \begin{bmatrix}
0 & 1 & 2 \\
1 & 0 & 1 \\
0 & 1 & 0
\end{bmatrix}$$1
$$B = B'P \rightarrow BB'^{-1} = P$$
$$\rightarrow \begin{bmatrix}
-1/2 & 1 & 1/2 \\
0 & 0 & 1 \\
1/2 & 0 & -1/2
\end{bmatrix}\begin{bmatrix}
1 & 2 & 3 \\
2 & 1 & 1 \\
0 & 2 & 1
\end{bmatrix} = P$$
$$\rightarrow P = \begin{bmatrix}
3/2 & 1 & 0 \\
0 & 2 & 1 \\
1/2 & 0 & 1
\end{bmatrix}$$
\item[(iv)]
$B$ must be invertible. So, $p$ and $\det B$ must be relatively prime. So, $B$ is a basis of $\mathbb{F}_p^3$ for all $p$ excluding $p = 7$.
\end{itemize}
\item[(6)]
The two bases are related by: $[B] = [B']P$. $[B']$ is invertible, so therefore $P = [B']^{-1}[B]$.
\item[(7)]
Note that each step corresponds to elementary matrices. There exists some $P$ such that $B = B'P$. Since $P$ is invertible, then we can write $P$ as a product of elementary matrices, proving the statement.
\item[(8)]
Since $L$ is in the span of $S$, then there exists a matrix $A$ such that $SA = L$. Let $U$ be a linear combination of vectors of $L$: $U = LC = SAC$. If $AC$ is the 0 matrix, then $U = 0$. If $AC = 0$ has a nontrivial solution (solving for $C$), then $L$ is linearly dependent. But if $|S| < |L|$, then $AC = 0$ has a nontrivial solution: therefore $|S| \geq |L|$.
\item[(9)]
Let $(v_1, ..., v_n)$ be a basis of $F^n$. Then, $\varphi((v_1, ..., v_n)) = [v_1 | ... | v_n]$. Clearly, $\varphi$ is injective, and surjective onto $GL_n(F)$.
\item[(10)]
$$\frac{(81^3 - 1)}{81 - 1} = 6643$$
\item[(11)]
\begin{itemize}
\item[(a)]
$$|SL_2(F)| = \frac{1}{p - 1}(p^2 - 1)(p^2 - p) = p(p^2 - 1)$$
\item[(b)]
$$|GL_n(F)| = (p^n - 1)(p^n - p)...(p^n - p^{n - 1})$$
$$|SL_n(f)| = \frac{1}{p - 1}|GL_n(F)|$$
$$|\mathcal{B}| = |GL_n(F)|$$
\end{itemize}
\item[(12)]
\begin{itemize}
\item[(a)]
Suppose $B$ is the left inverse of $A$. Note that if $n - m$ rows of zeros are added to the bottom of $A$ to form $A'$, then the left inverse $B'$ also contains $B$. But, since $B'$ does not exist, then $B$ also does not exist.

More concretely, suppose $B$ exists. Since $(BA)_{ii} = 1$, then now column of $A$ can be all 0s. Then, consider the first row of $BA$. The $i$th element of the first row is 
$$c_i = \sum_{j} a_{ji}b_{1j}$$
Since $c_2 = ... = c_n = 0$ with $m < n$, then $b_{11} = ... = b_{1m}$. But then $c_1 = 0$, a contradiction. Thus, $B$ cannot exist.
\item[(b)]
Define $P, P'$ such that $B = B'P', B' = BP$, ie. $P, P'$ are matrices of change of basis. So, $P'P = I, PP' = I$. So, $P' = P^{-1}, P = P'^{-1}$. Thus, $m = n$.
\end{itemize}
\end{itemize}
\end{document}