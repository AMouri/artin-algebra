\documentclass[12pt]{article}
\usepackage{amsmath, amssymb}
\begin{document}
\title{Chapter 2: Groups \\ Section 2: Subgroups}
\author{Alec Mouri}

\maketitle
\section*{Exercises}
\begin{itemize}
\item[(1)]
$$x = \begin{bmatrix}
1 & 1 \\
-1 & 0
\end{bmatrix}, x^2 = \begin{bmatrix}
0 & 1 \\
-1 & -1
\end{bmatrix}, x^3 = \begin{bmatrix}
-1 & 0 \\
0 & -1
\end{bmatrix} $$
$$x^4 = \begin{bmatrix}
-1 & -1 \\
1 & 0
\end{bmatrix}, x^5 = \begin{bmatrix}
0 & -1 \\
1 & 1
\end{bmatrix}, x^6 = 1 = \begin{bmatrix}
1 & 0 \\
0 & 1
\end{bmatrix}$$
\item[(2)]
$$a^3b = ba^3 \rightarrow a^3a^3b = a^3ba^3 \rightarrow ab = a^5ab = a^3b^3$$
And
$$a^3b = ba^3 \rightarrow a^3ba^3 = ba^3a^3 \rightarrow a^3ba^3 = baa^5 = ba$$
Thus, $ab = ba$
\item[(3)]
\begin{itemize}
\item[(a)] Yes. $GL_N(\mathbb{R})$ is a group, so it is a subgroup of $GL_N(\mathbb{C})$.
\item[(b)] Yes. Let $A, B \in \left\lbrace 1, -1 \right\rbrace$. Clearly, $AB \in \left\lbrace 1, -1 \right\rbrace$. And, $1 \in \left\lbrace 1, -1 \right\rbrace$, $1^{-1} = 1$, and $(-1)^{-1} = -1$. So, $\left\lbrace 1, -1 \right\rbrace$ is a subgroup of $\mathbb{R}^\times$.
\item[(c)] No. Let $\mathcal{A}$ be the set of positive integers in $\mathbb{Z}^+$. For $a \in \mathcal{A}$ where $a \neq 0$, $-a \not \in \mathcal{A}$. So $\mathcal{A}$ is not a subgroup.
\item[(d)]Yes. Let $\mathcal{A}$ be the set of positive reals in $\mathbb{R}^\times$. For $a, b \in \mathcal{A}$, clearly $ab \in \mathcal{A}$. And, $1 \in \mathcal{A}$. For $a \in \mathcal{A}$, since $a^{-1} > 0$, then $a^{-1} \in \mathcal{A}$. so $\mathcal{A}$ is a subgroup of $\mathbb{R}^\times$.
\item[(e)] No. Let $a = 1$. Then
$$\begin{bmatrix}
1 & 0 \\
0 & 0
\end{bmatrix} \not \in GL_2(\mathbb{R})$$
Since
$$\det\begin{bmatrix}
1 & 0 \\
0 & 0
\end{bmatrix} = 0$$
\end{itemize}
\item[(4)]
Let $x \in H$. Since $1 = xx^{-1} \in H$, then $H$ contains the identity. And, since $x^{-1} = 1x^{-1} \in H$, then $H$ is closed under inverses. Let $x, y \in H$. Then $y^{-1} \in H$. Then $xy = x(y^{-1})^{-1} \in H$. Thus, $H$ is a subgroup of $G$.
\item[(5)]
Let $\mathcal{A}$ be the $n$th roots of unity. Let $a, b \in \mathcal{A}$. Since $(ab)^n = a^nb^n = 1$, then $ab \in \mathcal{A}$. And, since $1^n = 1$, then $1 \in \mathcal{A}$. And, $(a^{-1})^n = (a^n)^{-1} = 1^{-1} = 1$. Thus $\mathcal{A} \subseteq \mathbb{C}^\times$. There are $n$ such roots: $z = e^{\frac{2\pi k}{n}}$ for $k = 0, ..., n - 1$. $\mathcal{A}$ is generated by $z = e^{\frac{2\pi}{n}}$, so $\mathcal{A}$ is a cyclic subgroup of order $n$.
\item[(6)]
\begin{itemize}
\item[(a)]
Let
$$a = \begin{bmatrix}
-1 & \\
& 1
\end{bmatrix}, b = \begin{bmatrix}
1 & \\
& -1
\end{bmatrix}$$
$a^2 = b^2 = I$, and $ab = ba$.
\item[(b)]
For each of the following sets the identity matrix is contained, they are closed under multiplication, and they are closed under inverses (in particular, the inverse of any matrix in the Klein four group is itself). Thus, the subgroups of the Klein four group are:
$$\left\lbrace \begin{bmatrix}
1 & \\
& 1
\end{bmatrix} \right\rbrace, \left\lbrace \begin{bmatrix}
\pm 1 & \\
& \pm 1
\end{bmatrix} \right\rbrace, \left\lbrace \begin{bmatrix}
1 & \\
& 1
\end{bmatrix}, \begin{bmatrix}
-1 & \\
& -1
\end{bmatrix} \right\rbrace$$
$$\left\lbrace \begin{bmatrix}
\pm 1 & \\
& 1
\end{bmatrix} \right\rbrace, \left\lbrace \begin{bmatrix}
1 & \\
& \pm 1
\end{bmatrix} \right\rbrace$$
\end{itemize}
\item[(7)]
\begin{itemize}
\item[(a)]
Let $ar + bs, ax + by \in a\mathbb{Z} + b\mathbb{Z}$. Then $ar + bs + ax + by = a(r + x) + b(s + y) \in a\mathbb{Z} + b\mathbb{Z}$.

Further, $0 = 0r + bs \in a\mathbb{Z} + b\mathbb{Z}$, and $ar + bs + a(-r) + b(-s) = 0$, so $a\mathbb{Z} + b\mathbb{Z}$ is closed under inverses. Thus, $a\mathbb{Z} + b\mathbb{Z}$ is a subgroup of $\mathbb{Z}^+$.
\item[(b)]
Let $c = ar + bs \in a\mathbb{Z} + b\mathbb{Z}$. Then 
$$c = ar + bs = s(b + 7a) + ar - 7as = s(b + 7a) + (r - 7s)a$$
Thus, $a\mathbb{Z} = b\mathbb{Z}$ is generated by $a$ and $b + 7a$.
\end{itemize}
\item[(8)]
\begin{tabular}{|c||c|c|c|c|}
\hline
& $1$ & $i$ & $j$ & $k$ \\
\hline
\hline
$1$ & $1$ & $i$ & $j$ & $k$ \\
\hline
$i$ & $i$ & $-1$ & $k$ & $-j$ \\
\hline
$j$ & $j$ & $-k$ & $-1$ & $i$ \\
\hline
$k$ & $k$ & $j$ & $-i$ & $-1$ \\
\hline
\end{tabular}
\item[(9)]
Let $x \in H$. Then we can write $x$ has a string of products of $a$ and $b$ and their inverses. Ie. $x = a^{a_1}b^{b_1}...a^{a_n}b^{b_n}$, where $a_n, b_n$ are integers. Since $ab = ba$ and hence $a^{-1}b^{-1} = b^{-1}a^{-1}$, $ab^{-1} = b^{-1}a$, and $a^{-1}b = ba^{-1}$, then we can also write $x = a^{a_1+...+a_n}b^{b_1+...+b_n} = a^{b_1 + ... + b-n}a^{a_1+...+a_n}$, or equivalently for some integers $c, d$, $x = a^cb^d = b^da^c$. 

So, let $x = a^{x_a}b^{x_b}, y = a^{y_a}b^{y_b} \in H$, where $x_a, x_b, y_a, y_b$ are integers. Then
$$xy = a^{x_a}b^{x_b}a^{y_a}b^{y_b} = a^{y_a}b^{y_b}a^{x_a}b^{x_b} = yx$$
Thus, $H$ is abelian.
\item[(10)]
\begin{itemize}
\item[(a)]
If $x$ has order $rs$, then $(x^r)^s = x^{rs} = 1$. Suppose for some $k < s$, $(x^r)^k = 1$. Then this implies $x^{rk} = 1$. But $rk < rs$, so $rs$ is not the order of $x$. Contradiction. Thus $s$ is the order of $x^r$.
\item[(b)]
If $x$ has order $n$, then for some $s$ and $k$ such that $rs = kn$, then $(x^r)^s = x^{rs} = x^{kn} = (x^n)^k = 1$. Choose $k$ such that $k = r/\text{gcd}(n, r)$. Then $r$ divides $kn$, so $s$ is an integer, namely $s = n/\text{gcd}(n,r)$. We now claim that $s$ is the order of $x^r$. Let $a$ be the order of $x^r$. Then $a$ divides $s$. Since $(x^r)^a = x^{ra} = 1$, then $n$ divides $ra$. Since $n/\text{gcd}(n,r)$ does not divide $r$ unless $\text{gcd}(n,r) = n$ or $r = 1$, then $s = n/\text{gcd}(n,r)$ divides $a$. Thus, $s = a$.
\end{itemize}
\item[(11)]
Let $|ab| = n$. Then $1 = (ab)^n = a(ba)^{n-1}b \rightarrow a^{-1}b^{-1} = (ba)^{n-1} \rightarrow (ba)^{-1} = (ba)^{n-1} \rightarrow 1 = (ba)^n$. Suppose now that $|ba| = m$, so that $n \geq m$. We can similarly show that $(ab)^m = 1$, so then $n \leq m$. Thus, $n = m$.
\item[(12)]
Clearly, the trivial group has no proper subgroup.

Suppose we have a nontrivial group $G$ with no proper subgroup. So for all $g \in G$ where $g \neq e$, where $e$ is the identity, $g$ generates $G$. Suppose $|G| = \infty$. Then $g^2$ generates a nontrivial subgroup, since it does not contain $g$. Thus, for $|G| < \infty$, $G = \left\lbrace e, g, g^2, ..., g^{|G| - 1} \right\rbrace$. Furthermore, for all $1 \leq k \leq |G| - 1$, $G = \left\lbrace e, g^k, (g^k)^2, ..., (g^k)^{|G|-1} \right\rbrace$, where for each $i$, $g^{ik} \neq 1$. Suppose $|G|$ is composite. Then for some $k$ and $a$, $|G| = ka$. Thus, $(g^k)^a = 1$. Thus, $|G|$ must be prime.

So, a nontrivial group $G$ with no proper subgroup is a cyclic subgroup with prime order.
\item[(13)]
Let $G$ be a cyclic group with generator $g$, and let $A$ be a subgroup of $G$. Then for some $a \in A, a = g^k$ for some $k$, and $a$ has order $n$. Suppose for some $b \in A$, $b$ is not generated by $a$, and $a$ is not generated by $b$. Then for some $\ell$, $b = g^\ell$, where $\text{gcd}(k, \ell) = 1$, and $b$ has order $m$. But since for some $c, d$, $g^{ck + d\ell} = g$, then $A = G$, so $A$ is cyclic. Thus, $b$ is generated by $a$, or $a$ is generated by $b$. Thus, $A$ is a cyclic group.
\item[(14)]
Suppose $G$ is generated by $g$, and $n = pr$ for some $p$. Note that $\mathcal{A} = \left\lbrace e, g^p, ..., g^{p(r-1)} \right\rbrace$ is a cyclic subgroup of $G$ of order $r$ generated by $g^p$. I claim that $\mathcal{A}$ is the only cyclic subgroup of $G$ of order $r$. Suppose there exists some other subgroup of $G$, $\mathcal{B}$, where the order of $\mathcal{B}$ is $r$, and $\mathcal{B}$ is generated by $g^q$ for some $q$. Then $(g^q)^r = g^{qr} = 1$. So $n$ divides $qr$, and so $p$ divides $q$. But then $g^p$ generates $g^q$. So, $\mathcal{B}$ is generated by $g^p$. Thus, $\mathcal{A} = \mathcal{B}$.
\item[(15)]
\begin{itemize}
\item[(a)]
Let $e$ be the identity of $H$ and $f$ be the identity of $G$. Then $e^2 = e = ef \rightarrow e = f$.
\item[(b)]
Let $a, b \in H, c \in G$, where $b$ is the inverse of $a$ in $H$, and $c$ is the inverse of $a$ in $G$. Then $e = ab = ac \rightarrow bab = bac \rightarrow b = c$.
\end{itemize}
\item[(16)]
\begin{itemize}
\item[(a)]
Let $G = \left\lbrace 1, g, g^2, g^3, g^4, g^5 \right\rbrace$. Since $1^1 = 1, g^6 = 1, (g^2)^3 = 1, (g^3)^2 = 1, (g^4)^3 = 1, (g^5)^6 = 1$, then $g$ and $g^5$ generate $G$
\item[(b)]
Let $A = \left\lbrace 1, a, a^2, a^3, a^4 \right\rbrace$. Since $1^1 = 1, a^5 = 1, (a^2)^5 = 1, (a^3)^5 = 1, (a^4)^5 = 1$, then $a, a^2, a^3, a^4$ generate $A$.

Let $B = \left\lbrace 1, b, b^2, b^3, b^4, b^5, b^6, b^7 \right\rbrace$. Since $1^1 = 1, b^8 = 1, (b^2)^4 = 1, (b^3)^8 = 1, (b^4)^2 = 1, (b^5)^8 = 1, (b^6)^4 = 1, (b^7)^8 = 1$, then $b, b^3, b^5, b^7$ generate $B$.

Let $C = \left\lbrace 1, c, c^2, c^3, c^4, c^5, c^6, c^7, c^8, c^9 \right\rbrace$. Since $1^1 = 1, c^{10} = 1, (c^2)^5 = 1, (c^3)^{10} = 1, (c^4)^5 = 1, (c^5)^2 = 1, (c^6)^5 = 1, (c^7)^{10} = 1, (c^8)^5 = 1, (c^9)^{10} = 1$, then $c, c^3, c^7, c^9$ generate $C$.
\item[(c)] If $g$ generates $G$ and has order $n$, then $g^k$ generates $G$ if $\text{gcd}(k, n) = 1$.
\end{itemize}
\item[(17)]
Let $a, b \in G$. Then $a^2 = b^2 = 1 \rightarrow a = a^{-1}, b = b^{-1}$. Furthermore, $(ab)^2 = 1 \rightarrow ab = (ab)^{-1}$. Then
$$a = a^{-1}, b = b^{-1} \rightarrow ab = a^{-1}b^{-1} = (ba)^{-1} = ba$$
\item[(18)]
\begin{itemize}
\item[(a)]
Let $A$ be an elementary matrix of the second kind whose operation is interchanging rows $i$ and $j$. Ie.
$$A = \begin{bmatrix}
1 \\
& \ddots \\
& & 0 & & 1 \\
& & & \ddots \\
& & 1 & & 0 \\
& & & & & \ddots \\
& & & & & & 1
\end{bmatrix}$$
Let
$$E_1 = \begin{bmatrix}
1 \\
& \ddots \\
& & 1 \\
& & & \ddots \\
& & & & -1 \\
& & & & & \ddots \\
& & & & & & 1
\end{bmatrix}$$
$$E_2 = \begin{bmatrix}
1 \\
& \ddots \\
& & 1 \\
& & & \ddots \\
& & -1 & & 1 \\
& & & & & \ddots \\
& & & & & & 1
\end{bmatrix}$$
$$E_3 = \begin{bmatrix}
1 \\
& \ddots \\
& & 1 & & 1 \\
& & & \ddots \\
& & & & 1 \\
& & & & & \ddots \\
& & & & & & 1
\end{bmatrix}$$
Ie. $E_1$ scales row $j$ by -1, $E_2$ sets row $j$ equal to row $j$ minus row $i$, and $E_3$ sets row $i$ equal to row $i$ plus row $j$. Then $A = E_1E_2E_3E_2$. So, we can write an elementary matrix of the second kind as a product of elementary matrices of the first and third kinds. So then we can generate any invertible matrix with elementary matrices of the first and third kinds.
\item[(b)]
Clearly, $SL_n(\mathbb{R}) \subseteq GL_n(\mathbb{R})$. Let $A, B \in SL_n(\mathbb{R})$. Since $\det(AB) = \det(A)\det(B) = 1$, then $AB \in SL_n(\mathbb{R})$. Since $I_n \in SL_n(\mathbb{R})$, and $\det(A^{-1}) = \det(A)^{-1} = 1$, then $A^{-1} \in SL_n(\mathbb{R})$. Thus, $SL_n(\mathbb{R})$ is a subgroup of $GL_n(\mathbb{R})$. 
\item[(c)]
Consider the $2 \times 2$ matrix
$$A = \begin{bmatrix}
a & b \\
c & d
\end{bmatrix}$$
Since $\det A = 1$, then $ad - bc = 1$.

If $c \neq 0$, then reducing $A$:
$$\begin{bmatrix}
a & b \\
c & d
\end{bmatrix} \rightarrow \begin{bmatrix}
1 & b + d(1-a)/c \\
c & d
\end{bmatrix} \rightarrow \begin{bmatrix}
1 & b + d(1-a)/c \\
& ad - bc
\end{bmatrix} \rightarrow \begin{bmatrix}
1 & \\
& 1
\end{bmatrix} $$

If $c = 0$, then $a \neq 0$ (otherwise $\det A = 0$). Then:
$$\begin{bmatrix}
a & b \\
& d
\end{bmatrix} \rightarrow \begin{bmatrix}
a & b \\
1 - a & d + b(1 - a)/a
\end{bmatrix} \rightarrow \begin{bmatrix}
1 & d + b/a \\
1 - a & d + b(1 - a)/a
\end{bmatrix}$$
$$\rightarrow \begin{bmatrix}
1 & d + b/a \\
& ad
\end{bmatrix} \rightarrow \begin{bmatrix}
1 & \\
& 1
\end{bmatrix}$$
$A$ was reduced only with type 1 operations, so we can write $A$ as a product of elementary matrices of the first kind. 

Suppose for all $X \in SL_{n-1}(\mathbb{R})$ we can write $X$ as a product of elementary matrices of the first kind. Now consider an $n \times n$ matrix $B$. Suppose $b_{21} = ... = b_{n1} = 0$. Then $b_{11} \neq 0$ (otherwise $\det B = 0$). The following operations set $b_{11} = 1$ while keeping $b_{21} = ... = b_{n1} = 0$:
$$ \begin{bmatrix}
1 & 1 \\
& 1 \\
& & \ddots \\
& & & 1
\end{bmatrix}\begin{bmatrix}
1 \\
(1-b_1)/b_1 & 1 \\
& & \ddots \\
& & & 1
\end{bmatrix}$$
Suppose now that $b_{11} = 0$. Then for some $b_{i1}$, $b_{i1} \neq 0$. Then the following operation sets $b_{11} = 1$:
$$\begin{bmatrix}
1 & & 1/b_{i1} \\
& \ddots \\
& & 1 \\
& & & \ddots \\
& & & & 1
\end{bmatrix}$$
If $b_{11} = 1$, then we can perform type 1 operations to set $b_{21} = ... = b_{n1} = 0$: namely if $b_{i1} \neq 0$, then we can perform the following operation:
$$\begin{bmatrix}
1 \\
& \ddots \\
-b_{i1} & & 1 \\
& & & \ddots \\
& & & & 1
\end{bmatrix}$$
After applying the above operations, now $b_{11} = 1$ and $b_{21} = ... = b_{n1} = 0$. Consider the submatrix $B_{11}$. Since $1 = \det B = b_{11}\det B_{11}$, then $\det B_{11} = 1$, since $B_{11} \in SL_{n-1}(\mathbb{R})$. By the inductive hypothesis, there exists a series of type 1 matrices such that $B_{11}$ can be reduced to the identity. So, $B$ can be reduced to an upper triangular matrix $B'$ with 1s as its diagonal entries. Now we can apply type one operations to clear the entries above the diagonal, so that $B'$ is row reduced to the identity. Since for elementary matrices of the first kind $E_1, ..., E_p$ exist such that $E_1...E_pB = I$, then $B = E_p^{-1}...E_1^{-1}$ is a product of elementary matrices of the first kind. Thus elementary matrices of the first kind generate $SL_n(\mathbb{R})$.
\end{itemize}
\item[(19)]
$$\begin{bmatrix}
& 1 \\
1 & \\
& & 1 \\
& & & 1
\end{bmatrix}, \begin{bmatrix}
& & 1 \\
& 1 \\
1 \\
& & & 1
\end{bmatrix}, \begin{bmatrix}
& & & 1 \\
& 1 \\
& & 1 \\
1
\end{bmatrix}$$
$$\begin{bmatrix}
1 \\
& & 1 \\
& 1 \\
& & & 1
\end{bmatrix}, \begin{bmatrix}
1 \\
& & & 1 \\
& & 1 \\
& 1
\end{bmatrix}, \begin{bmatrix}
1 \\
& 1 \\
& & & 1 \\
& & 1
\end{bmatrix}$$
$$\begin{bmatrix}
& 1 \\
1 \\
& & & 1 \\
& & 1
\end{bmatrix}, \begin{bmatrix}
& & 1 \\
& & & 1 \\
1 \\
& 1
\end{bmatrix}, \begin{bmatrix}
& & & 1 \\
& & 1 \\
& 1 \\
1
\end{bmatrix}$$
There are 9 elements with order 2 in $S_4$.
\item[(20)]
\begin{itemize}
\item[(a)]
Note that 
$$(ab)^{nm/\text{gcd}(n,m)} = a^{nm/\text{gcd}(n,m)}b^{nm/\text{gcd}(n,m)}$$
$$= (a^m)^{n/\text{gcd}(n,m)}(b^n)^{m/\text{gcd}(n,m)} = 1$$ So $ab$ has finite order as most $\frac{nm}{\text{gcd}(n,m)}$. 
\item[(b)]
Consider
$$A = \begin{bmatrix}
0 & 1 \\
-1 & 0
\end{bmatrix}, B = \begin{bmatrix}
0 & -1 \\
1 & 1
\end{bmatrix}$$
Note that $A^4 = I, B^6 = I$. But
$$AB = \begin{bmatrix}
1 & 1 \\
& 1
\end{bmatrix}, (AB)^n = \begin{bmatrix}
1 & n \\
& 1
\end{bmatrix}$$
So, $AB$ does not have finite order.
\end{itemize}
\item[(21)]
From the previous exercise, then if $a, b \in G$, where $G$ is an abelian group and $a, b$ have finite order, then $ab$ also has finite order. Let $H$ be the subset of $G$ whose elements have finite order. Clearly, $e \in H$, where $e$ is the identity. And, for $a \in H$, where $|a| = n$, then $a^{-1} = a^{n-1}$. So, $H$ is a subgroup of $G$.
\item[(22)]
Let $a = p_1^{a_1}...p_n^{a_n}, b = p_1^{b_1}...p_n^{b_n}$, where $p_1...p_n$ are primes and $a_1, ..., a_n, b_1, ..., b_n \geq 0$. Such product of primes is unique by the Fundamental Theorem of Arithmetic. Let $g$ be the greatest common divisor of $a$ and $b$. Since $g$ divides $a$ and $g$ divides $b$, then $g = p_1^{c_1}...p_n^{c_n}$, where for all $i$, $c_i \geq 0, c_i \leq a_i, c_i \leq b_i$. Suppose some integer $h$ also divides $a$ and $b$. Then $h = p_1^{d_1}...p_n^{d_n}$. Then $g$ also divides $h$. So for all $i$, $c_i \geq d_i$, ie. $c_i$ is the largest integer such that $c_i \leq a_i$ and $c_i \leq b_i$. Thus, $g = p_1^{\min(a_1,b_1)}...p_n^{\min(a_n,b_n)}$.
\end{itemize}
\end{document}