\documentclass[12pt]{article}
\usepackage{amsmath, amssymb}
\begin{document}
\title{Chapter 2: Groups \\ Section 9: Modular Arithmetic}
\author{Alec Mouri}

\maketitle`
\section*{Exercises}
\begin{itemize}
\item[(1)]
$$(7 + 14)(3 - 16)\equiv (21)(-13) \equiv (4)(4) \equiv 16 \mod 17$$
\item[(2)]
\begin{itemize}
\item[(a)]
$$0^2 \equiv 0 \mod 4$$
$$1^2 \equiv 1 \mod 4$$
$$2^2 \equiv 4 \equiv 0 \mod 4$$
$$3^2 \equiv 9 \equiv 1 \mod 4$$
So $a^2 \mod 4$ is always either 0 or 1.
\item[(b)]
$$0^2 \equiv 0 \mod 8$$
$$1^2 \equiv 1 \mod 8$$
$$2^2 \equiv 4 \mod 8$$
$$3^2 \equiv 9 \equiv 1 \mod 8$$
$$4^2 \equiv 16 \equiv 0 \mod 8$$
$$5^2 \equiv 25 \equiv 1 \mod 8$$
$$6^2 \equiv 36 \equiv 4 \mod 8$$
$$7^2 \equiv 49 \equiv 1 \mod 8$$
So $a^2 \mod 8$ is always either 0, 1, or 4.
\end{itemize}
\item[(3)]
\begin{itemize}
\item[(a)]
$$(0)(2) \equiv 0 \mod 6$$
$$(1)(2) \equiv 2 \mod 6$$
$$(2)(2) \equiv 4 \mod 6$$
$$(3)(2) \equiv 6 \equiv 0 \mod 6$$
$$(4)(2) \equiv 8 \equiv 2 \mod 6$$
$$(5)(2) \equiv 10 \equiv 4 \mod 6$$
So, $2$ has no inverse modulo 6.
\item[(b)]
Suppose $a$ is the inverse of 2 modulo $n$. Then for some $b \in \mathbb{Z}$,
$$1 = 2a + bn \rightarrow n = (1 - 2a)b^{-1}$$
Since $1 - 2a$ is not even, then $n$ is odd.
\end{itemize}
\item[(4)]
Note that $(10)^i \equiv 1^i \equiv 1 \mod 9$. So,
$$a = d_0 + 10d_1 + ... + 10^nd_n \equiv d_0 + d_1 + ... + d_n \mod 9$$
\item[(5)]
\begin{itemize}
\item[(a)]
$$x = 2^{-1}5 \equiv (5)(5) \equiv 7 \mod 9$$
\item[(b)]
$$2x \equiv 5$$ has no solution modulo 6, since the possible values of $2x$ are $0, 2, 4$ modulo 6.
\end{itemize}
\item[(6)]
For all $n$, $x + y \equiv 2$ has a solution: $x = y = 1$. Note if $n = 1$, then $0 \equiv 1 \equiv 2$ modulo $n$, so the statement is trivially true.

If $2x - 3y \equiv 3 \mod n$, then $2x - 3y - 3$ divides $n$. If $x = 0, y = -1$, then $2x - 3y - 3 = 0$. 0 divides all integers, so the statement has a solution for all $n$.
\item[(7)]
$$(\overline{a}\cdot\overline{b})\cdot\overline{c} = ((a + cn)(b + dn))(c + en)$$
$$= (a + cn)((b + dn)(c + en)) = \overline{a}\cdot(\overline{b}\cdot\overline{c})$$

$$\overline{a}\cdot\overline{b} = (a + cn)(b + dn) = (b + dn)(a + cn) = \overline{b}\cdot\overline{a}$$
\item[(8)]
From Proposition 2.6, $1 = an + bm$. Then
$$1 \equiv an + bm \mod m \rightarrow 1 \equiv an \mod m$$
Since $gcd(n, m) = 1$, then $n^{-1} \equiv a \mod m$.

Similarly, $m^{-1} \equiv b \mod n$.
\end{itemize}
\end{document}